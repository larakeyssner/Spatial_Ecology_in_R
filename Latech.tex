\documentclass[12pt]{article}
\usepackage{graphicx} % (%) for comments --> Required for inserting images liek package in R 
\usepackage{hyperref} %to make reffernces package 
\usepackage{natbib} %package for citation

\title{This is my first LaTeX document} 
\author{Duccio Rocchini}
% \date{} %if procentage before it will autmoatically take date from the computer so code not read but directely taken 

\begin{document} %and at the end of document the end fucntion needed 

\maketitle

\linenumbers

\tableofcontents

\section{Introduction}
\section{Methods}
\section{Results}
\section{Discussion}
\section*{Appendix} %so no number before 

\begin{abstract}
Let me take you down
'Cause I'm going to strawberry fields
Nothing is real
And nothing to get hung about
Strawberry fields forever
Living is easy with eyes closed
Misunderstanding all you see
It's getting hard to be someone, but it all works out
It doesn't matter much to me
Let me take you down
'Cause I'm going to strawberry fields
Nothing is real
And nothing to get hung about

\noindent so when the new text starts no gap a front 
\smallskip then text to have a little gap 
\bigskip to make the skip bigger 

\end{abstract}

\section{Introduction}

Let me take you down
'Cause I'm going to strawberry fields
Nothing is real
And nothing to get hung about


% \smallskip
\bigskip
Let me take you down
'Cause I'm going to strawberry fields
Nothing is real
And nothing to get hung about
Strawberry fields forever



\section{Methods}
Here functions for gravity 

in the text there is considered the Equation \ref{eq:newton}and then the complex Equation \ref{eq:complex}

%to make relations reffernces to lables (lo longer with numbers so if we change something its not all messed up 

\begin{equation}
    F = G \times\frac{m_1\times m_2}{r^2}
    \label{eq:newton}
\end{equation}


\begin{equation}
    F = \sqrt[3]{G \times\frac{m_1\times m_2}{r^2}}
    \label{eq:complex}
\end{equation}

%check wikibooks to see the right functions 


Concerning the algorithm being used, we relied on Equation \ref{newton}:

\begin{equation}
    F = \sqrt[3]{G \times \frac{m_{1444} \times m_2}{r^{21111}}}
    \label{newton}
\end{equation}


Let me take you down
'Cause I'm going to strawberry fields
Nothing is real
And nothing to get hung about
Strawberry fields forever
Living is easy with eyes closed
Misunderstanding all you see
It's getting hard to be someone, but it all works out
It doesn't matter much to me
Let me take you down
'Cause I'm going to strawberry fields
Nothing is real
And nothing to get hung about


The main steps followed in this project were based on:
\begin{itemize}
    \item Data gathering
    \item Data analysis
    \item Creation of graphical output
\end{itemize}

The main steps followed in this project were based on:
\begin{enumerate}
    \item Data gathering
    \item Data analysis
    \item Creation of graphical output
\end{enumerate}

\section{Results}

Here can find the graphs 



The main result can be seen in Figure \ref{output}.
%first have to upload the figure and then it can be used 
\begin{figure}
    \centering
    \includegraphics[width=\linewidth]{ggplot_1_single_smooth.png}
    \caption{Trend of the analysed data...}
    \label{output}
\end{figure}

\section{Discussion}

And nothing to get hung about \cite{Massol2011}

Strawberry fields forever \cite{Anderson2013}.

\begin{thebibliography}{999}
    \bibitem[Anderson and Geston (2013)]{Anderson2013}
    Anderson, Karen, and Kevin J. Gaston. "Lightweight unmanned aerial vehicles will revolutionize spatial ecology." Frontiers in Ecology and the Environment 11.3 (2013): 138-146.
%
    \bibitem[Massol (2011)]{Massol2011}
    Massol, François, et al. "Linking community and ecosystem dynamics through spatial ecology." Ecology letters 14.3 (2011): 313-323.
\end{thebibliography}


%the label for the cite is {Massol2011}

\end{document}
